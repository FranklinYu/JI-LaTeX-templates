%% File: example.tex
%%
%% This is an example for class `Ve270lab'.
%%
\documentclass{Ve270lab}

\labnumber{1}
\student{Franklin}
\experiment{Fantastic Experiment}
\coursetitle{Great Course}
\university{Marvelous University}

\begin{document}

\section{Objectives}
What are you designing?

\section{Problem Definition}
Define functional and timing specifications of the system/component under design. Define I/O interfaces of the system/component.



\section{System Partitioning}
Describe how system is partitioned. Describe how each subsystem is functioning and interacting with other subsystems.

\emph{Do NOT copy from the lab manual.}

\section{Design Entry}
Include hierarchical design entries: Schematics, truth tables, state machines etc.

\section{Test Plan}
A good test plan describes what to test and how to test them, preferably in a tabular manner. Develop a test plan for the entire system and each subsystem if necessary, so that you show readers that your design is behaving according to the requirements.

\section{Simulation Results}
Test results are displayed here. Do not put any figure without a reference in your text.

\section{Hardware Implementation and Testing}
Logic synthesis, FPGA implementation, and board testing. Describe how your design is tested on the hardware implementation.

\section{Conclusions}
Conclude your report here. Report any problem and proposed solutions here as well.

\section*{Appendix}
\addcontentsline{toc}{section}{Appendix}
Schematics and/or Verilog HDL codes for your design.

\end{document}
